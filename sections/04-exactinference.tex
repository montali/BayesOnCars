\section{Exact inference}
\label{sec:exactinference}
\subsection{Theoretical background}
\subsubsection{Summing out}
At this moment, it is very well clear that the basic task for a probabilistic inference system would be, as the word says, performing \textbf{inference}, i.e. computing the posterior probability for a set of query variables, given some observed event in the form of evidence variable. Moreover, it is clear that by \textbf{summing out} the probabilities from the full joint distribution, we are able to compute any conditional probability. More specifically, a query $\mathbf{P}(X \mid \mathbf{e})$ can be always computed as:
\begin{equation}
    \mathbf{P}(X \mid \mathbf{e})=\alpha \mathbf{P}(X, \mathbf{e})=\alpha \sum_{\mathbf{y}} \mathbf{P}(X, \mathbf{e}, \mathbf{y})
\end{equation}
\subsubsection{Variable elimination}
A crucial improvement in this algorithm is provided by \textbf{variable elimination}, which we can simplify as a form of \textbf{dynamic programming}, in which the factors get saved into an array to avoid recomputing the same thing multiple times.
\subsection{Variable elimination with pgmpy}
As always, \texttt{pgmpy} allows us to do so in a straightforward way:
\begin{minted}{python}
from pgmpy.inference import VariableElimination
exact_inference = VariableElimination(bayesian_model)
\end{minted}
Having instantiated the \texttt{exact\_inference} object, we can then call queries on it. For example, a nice question to ask would be:
\begin{center}
    \textit{Is maintenance harder for 2-seaters than for family cars?}
\end{center}
With just two lines of code, we can produce the probability tables for this query:
\begin{minted}{python}
print(exact_inference.query(["Maintenance"],{'People':"2"}))
print(exact_inference.query(["Maintenance"],{'People':"4"}))
\end{minted}
\begin{center}
\begin{figure}[H]
\subfloat[Maintenance on 2-seaters]{
    \begin{tabular}{|l|l|}
        \hline
        \textbf{Maintenance level}&\textbf{Probability}\\
        \hline
        very high & 0.2975 \\
        \hline
        high & 0.2595 \\
        \hline
        medium & 0.2215 \\
        \hline
        low & 0.2215 \\
        \hline
    \end{tabular}
}
\subfloat[Maintenance on 4-seaters]{
    \begin{tabular}{|l|l|}
        \hline
        \textbf{Maintenance level}&\textbf{Probability}\\
        \hline
        very high & 0.2256 \\
        \hline
        high & 0.2450 \\
        \hline
        medium & 0.2646 \\
        \hline
        low & 0.2648 \\
        \hline
    \end{tabular}
}
\end{figure}
\end{center}
It looks like sport cars are not the best kind of choice if you want to stay cheap on oil and filters.
\begin{center}
    \textit{Are cars with high maintenance safe?}
    \end{center}
\begin{center}
    \begin{tabular}{|l|l|}
        \hline
        \textbf{Safety}&\textbf{Probability}\\
        \hline
        high & 0.2793 \\
        \hline
        medium & 0.3240 \\
        \hline
        low & 0.3967 \\
        \hline
    \end{tabular}
    \end{center}

It looks like they are not!
