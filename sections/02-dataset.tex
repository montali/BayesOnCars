\section{Dataset}
\label{sec:dataset}
This dataset \href{https://archive.ics.uci.edu/ml/datasets/Car+Evaluation}{(click here)} was created in 1997 by two researchers of the Jožef Stefan Institute in Ljubljana, Slovenija to demonstrate a qualitative multi-criteria decision analysis (MCDA) method for decision making, DEX. The provided data consists of the classification of different features in $\approx 1800$ cars. The target attribute is the \textbf{car acceptability}, a measure of the interest in the car demonstrated by the public. It is obvious that this dataset is a strong simplification of what actually happens when buying a car, and the fact that it was artificially crafted by someone doesn't help. Nonetheless, it perfectly fits the needs of a project like this one. The features that are stated for each car are the following: the buying price, the maintenance price, the number of doors, the seats, the size of the lug boot, the safety. Obviously, the numeric attributes like the prices are \textbf{bucketized}. Therefore, the final attribute list is the following:
\begin{center}
\begin{tabular}{|l|l|}
\hline
\textbf{attribute} & \textbf{values}\\
\hline
acceptability & \textit{unacc, acc, good, vgood}\\
\hline
buying & \textit{vhigh, high, med, low}\\
\hline
maint & \textit{vhigh, high, med, low}\\
\hline
doors & \textit{2,3,4,5more}\\
\hline
people & \textit{2,4,more}\\
\hline
lug\_boot & \textit{small, med, big}\\
\hline
safety & \textit{low, med, high}\\
\hline
\end{tabular}
\end{center}
The data is in CSV format, so a car in the dataset looks like this:
\begin{center}
    \texttt{vhigh,vhigh,3,2,small,high,unacc}
\end{center}